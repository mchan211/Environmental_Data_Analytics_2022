% Options for packages loaded elsewhere
\PassOptionsToPackage{unicode}{hyperref}
\PassOptionsToPackage{hyphens}{url}
%
\documentclass[
]{article}
\title{Assignment 5: Data Visualization}
\author{Meilin Chan}
\date{}

\usepackage{amsmath,amssymb}
\usepackage{lmodern}
\usepackage{iftex}
\ifPDFTeX
  \usepackage[T1]{fontenc}
  \usepackage[utf8]{inputenc}
  \usepackage{textcomp} % provide euro and other symbols
\else % if luatex or xetex
  \usepackage{unicode-math}
  \defaultfontfeatures{Scale=MatchLowercase}
  \defaultfontfeatures[\rmfamily]{Ligatures=TeX,Scale=1}
\fi
% Use upquote if available, for straight quotes in verbatim environments
\IfFileExists{upquote.sty}{\usepackage{upquote}}{}
\IfFileExists{microtype.sty}{% use microtype if available
  \usepackage[]{microtype}
  \UseMicrotypeSet[protrusion]{basicmath} % disable protrusion for tt fonts
}{}
\makeatletter
\@ifundefined{KOMAClassName}{% if non-KOMA class
  \IfFileExists{parskip.sty}{%
    \usepackage{parskip}
  }{% else
    \setlength{\parindent}{0pt}
    \setlength{\parskip}{6pt plus 2pt minus 1pt}}
}{% if KOMA class
  \KOMAoptions{parskip=half}}
\makeatother
\usepackage{xcolor}
\IfFileExists{xurl.sty}{\usepackage{xurl}}{} % add URL line breaks if available
\IfFileExists{bookmark.sty}{\usepackage{bookmark}}{\usepackage{hyperref}}
\hypersetup{
  pdftitle={Assignment 5: Data Visualization},
  pdfauthor={Meilin Chan},
  hidelinks,
  pdfcreator={LaTeX via pandoc}}
\urlstyle{same} % disable monospaced font for URLs
\usepackage[margin=2.54cm]{geometry}
\usepackage{color}
\usepackage{fancyvrb}
\newcommand{\VerbBar}{|}
\newcommand{\VERB}{\Verb[commandchars=\\\{\}]}
\DefineVerbatimEnvironment{Highlighting}{Verbatim}{commandchars=\\\{\}}
% Add ',fontsize=\small' for more characters per line
\usepackage{framed}
\definecolor{shadecolor}{RGB}{248,248,248}
\newenvironment{Shaded}{\begin{snugshade}}{\end{snugshade}}
\newcommand{\AlertTok}[1]{\textcolor[rgb]{0.94,0.16,0.16}{#1}}
\newcommand{\AnnotationTok}[1]{\textcolor[rgb]{0.56,0.35,0.01}{\textbf{\textit{#1}}}}
\newcommand{\AttributeTok}[1]{\textcolor[rgb]{0.77,0.63,0.00}{#1}}
\newcommand{\BaseNTok}[1]{\textcolor[rgb]{0.00,0.00,0.81}{#1}}
\newcommand{\BuiltInTok}[1]{#1}
\newcommand{\CharTok}[1]{\textcolor[rgb]{0.31,0.60,0.02}{#1}}
\newcommand{\CommentTok}[1]{\textcolor[rgb]{0.56,0.35,0.01}{\textit{#1}}}
\newcommand{\CommentVarTok}[1]{\textcolor[rgb]{0.56,0.35,0.01}{\textbf{\textit{#1}}}}
\newcommand{\ConstantTok}[1]{\textcolor[rgb]{0.00,0.00,0.00}{#1}}
\newcommand{\ControlFlowTok}[1]{\textcolor[rgb]{0.13,0.29,0.53}{\textbf{#1}}}
\newcommand{\DataTypeTok}[1]{\textcolor[rgb]{0.13,0.29,0.53}{#1}}
\newcommand{\DecValTok}[1]{\textcolor[rgb]{0.00,0.00,0.81}{#1}}
\newcommand{\DocumentationTok}[1]{\textcolor[rgb]{0.56,0.35,0.01}{\textbf{\textit{#1}}}}
\newcommand{\ErrorTok}[1]{\textcolor[rgb]{0.64,0.00,0.00}{\textbf{#1}}}
\newcommand{\ExtensionTok}[1]{#1}
\newcommand{\FloatTok}[1]{\textcolor[rgb]{0.00,0.00,0.81}{#1}}
\newcommand{\FunctionTok}[1]{\textcolor[rgb]{0.00,0.00,0.00}{#1}}
\newcommand{\ImportTok}[1]{#1}
\newcommand{\InformationTok}[1]{\textcolor[rgb]{0.56,0.35,0.01}{\textbf{\textit{#1}}}}
\newcommand{\KeywordTok}[1]{\textcolor[rgb]{0.13,0.29,0.53}{\textbf{#1}}}
\newcommand{\NormalTok}[1]{#1}
\newcommand{\OperatorTok}[1]{\textcolor[rgb]{0.81,0.36,0.00}{\textbf{#1}}}
\newcommand{\OtherTok}[1]{\textcolor[rgb]{0.56,0.35,0.01}{#1}}
\newcommand{\PreprocessorTok}[1]{\textcolor[rgb]{0.56,0.35,0.01}{\textit{#1}}}
\newcommand{\RegionMarkerTok}[1]{#1}
\newcommand{\SpecialCharTok}[1]{\textcolor[rgb]{0.00,0.00,0.00}{#1}}
\newcommand{\SpecialStringTok}[1]{\textcolor[rgb]{0.31,0.60,0.02}{#1}}
\newcommand{\StringTok}[1]{\textcolor[rgb]{0.31,0.60,0.02}{#1}}
\newcommand{\VariableTok}[1]{\textcolor[rgb]{0.00,0.00,0.00}{#1}}
\newcommand{\VerbatimStringTok}[1]{\textcolor[rgb]{0.31,0.60,0.02}{#1}}
\newcommand{\WarningTok}[1]{\textcolor[rgb]{0.56,0.35,0.01}{\textbf{\textit{#1}}}}
\usepackage{graphicx}
\makeatletter
\def\maxwidth{\ifdim\Gin@nat@width>\linewidth\linewidth\else\Gin@nat@width\fi}
\def\maxheight{\ifdim\Gin@nat@height>\textheight\textheight\else\Gin@nat@height\fi}
\makeatother
% Scale images if necessary, so that they will not overflow the page
% margins by default, and it is still possible to overwrite the defaults
% using explicit options in \includegraphics[width, height, ...]{}
\setkeys{Gin}{width=\maxwidth,height=\maxheight,keepaspectratio}
% Set default figure placement to htbp
\makeatletter
\def\fps@figure{htbp}
\makeatother
\setlength{\emergencystretch}{3em} % prevent overfull lines
\providecommand{\tightlist}{%
  \setlength{\itemsep}{0pt}\setlength{\parskip}{0pt}}
\setcounter{secnumdepth}{-\maxdimen} % remove section numbering
\ifLuaTeX
  \usepackage{selnolig}  % disable illegal ligatures
\fi

\begin{document}
\maketitle

\hypertarget{overview}{%
\subsection{OVERVIEW}\label{overview}}

This exercise accompanies the lessons in Environmental Data Analytics on
Data Visualization

\hypertarget{directions}{%
\subsection{Directions}\label{directions}}

\begin{enumerate}
\def\labelenumi{\arabic{enumi}.}
\tightlist
\item
  Change ``Student Name'' on line 3 (above) with your name.
\item
  Work through the steps, \textbf{creating code and output} that fulfill
  each instruction.
\item
  Be sure to \textbf{answer the questions} in this assignment document.
\item
  When you have completed the assignment, \textbf{Knit} the text and
  code into a single PDF file.
\item
  After Knitting, submit the completed exercise (PDF file) to the
  dropbox in Sakai. Add your last name into the file name (e.g.,
  ``Fay\_A05\_DataVisualization.Rmd'') prior to submission.
\end{enumerate}

The completed exercise is due on Monday, February 14 at 7:00 pm.

\hypertarget{set-up-your-session}{%
\subsection{Set up your session}\label{set-up-your-session}}

\begin{enumerate}
\def\labelenumi{\arabic{enumi}.}
\item
  Set up your session. Verify your working directory and load the
  tidyverse and cowplot packages. Upload the NTL-LTER processed data
  files for nutrients and chemistry/physics for Peter and Paul Lakes
  (use the tidy
  {[}\texttt{NTL-LTER\_Lake\_Chemistry\_Nutrients\_PeterPaul\_Processed.csv}{]}
  version) and the processed data file for the Niwot Ridge litter
  dataset (use the
  {[}\texttt{NEON\_NIWO\_Litter\_mass\_trap\_Processed.csv}{]} version).
\item
  Make sure R is reading dates as date format; if not change the format
  to date.
\end{enumerate}

\begin{Shaded}
\begin{Highlighting}[]
\CommentTok{\#1 }
\FunctionTok{getwd}\NormalTok{()}
\end{Highlighting}
\end{Shaded}

\begin{verbatim}
## [1] "C:/Users/meili/OneDrive - Duke University/EDA/Environmental_Data_Analytics_2022/Assignments"
\end{verbatim}

\begin{Shaded}
\begin{Highlighting}[]
\FunctionTok{library}\NormalTok{(tidyverse)}
\end{Highlighting}
\end{Shaded}

\begin{verbatim}
## -- Attaching packages --------------------------------------- tidyverse 1.3.1 --
\end{verbatim}

\begin{verbatim}
## v ggplot2 3.3.5     v purrr   0.3.4
## v tibble  3.1.6     v dplyr   1.0.7
## v tidyr   1.1.4     v stringr 1.4.0
## v readr   2.1.1     v forcats 0.5.1
\end{verbatim}

\begin{verbatim}
## -- Conflicts ------------------------------------------ tidyverse_conflicts() --
## x dplyr::filter() masks stats::filter()
## x dplyr::lag()    masks stats::lag()
\end{verbatim}

\begin{Shaded}
\begin{Highlighting}[]
\FunctionTok{library}\NormalTok{(cowplot)}

\NormalTok{PP\_Nutrients }\OtherTok{\textless{}{-}} \FunctionTok{read.csv}\NormalTok{(}\StringTok{"../Data/Processed/NTL{-}LTER\_Lake\_Chemistry\_Nutrients\_PeterPaul\_Processed.csv"}\NormalTok{, }\AttributeTok{stringsAsFactors =} \ConstantTok{FALSE}\NormalTok{)}

\NormalTok{NN\_Litter }\OtherTok{\textless{}{-}} \FunctionTok{read.csv}\NormalTok{(}\StringTok{"../Data/Processed/NEON\_NIWO\_Litter\_mass\_trap\_Processed.csv"}\NormalTok{, }\AttributeTok{stringsAsFactors =} \ConstantTok{TRUE}\NormalTok{)}

\CommentTok{\#2 }
\FunctionTok{class}\NormalTok{(PP\_Nutrients}\SpecialCharTok{$}\NormalTok{sampledate)}
\end{Highlighting}
\end{Shaded}

\begin{verbatim}
## [1] "character"
\end{verbatim}

\begin{Shaded}
\begin{Highlighting}[]
\NormalTok{PP\_Nutrients}\SpecialCharTok{$}\NormalTok{sampledate }\OtherTok{\textless{}{-}} \FunctionTok{as.Date}\NormalTok{(PP\_Nutrients}\SpecialCharTok{$}\NormalTok{sampledate, }
                                   \AttributeTok{format =} \StringTok{"\%Y{-}\%m{-}\%d"}\NormalTok{)}
\FunctionTok{class}\NormalTok{(PP\_Nutrients}\SpecialCharTok{$}\NormalTok{sampledate)}
\end{Highlighting}
\end{Shaded}

\begin{verbatim}
## [1] "Date"
\end{verbatim}

\begin{Shaded}
\begin{Highlighting}[]
\FunctionTok{class}\NormalTok{(NN\_Litter}\SpecialCharTok{$}\NormalTok{collectDate)}
\end{Highlighting}
\end{Shaded}

\begin{verbatim}
## [1] "factor"
\end{verbatim}

\begin{Shaded}
\begin{Highlighting}[]
\NormalTok{NN\_Litter}\SpecialCharTok{$}\NormalTok{collectDate }\OtherTok{\textless{}{-}} \FunctionTok{as.Date}\NormalTok{(NN\_Litter}\SpecialCharTok{$}\NormalTok{collectDate, }
                                 \AttributeTok{format =} \StringTok{"\%Y{-}\%m{-}\%d"}\NormalTok{)}
\FunctionTok{class}\NormalTok{(NN\_Litter}\SpecialCharTok{$}\NormalTok{collectDate)}
\end{Highlighting}
\end{Shaded}

\begin{verbatim}
## [1] "Date"
\end{verbatim}

\hypertarget{define-your-theme}{%
\subsection{Define your theme}\label{define-your-theme}}

\begin{enumerate}
\def\labelenumi{\arabic{enumi}.}
\setcounter{enumi}{2}
\tightlist
\item
  Build a theme and set it as your default theme.
\end{enumerate}

\begin{Shaded}
\begin{Highlighting}[]
\CommentTok{\#3}
\NormalTok{A05\_theme }\OtherTok{\textless{}{-}} \FunctionTok{theme\_bw}\NormalTok{(}\AttributeTok{base\_size =} \DecValTok{12}\NormalTok{) }\SpecialCharTok{+}
  \FunctionTok{theme}\NormalTok{(}\AttributeTok{axis.text =} \FunctionTok{element\_text}\NormalTok{(}\AttributeTok{color =} \StringTok{"black"}\NormalTok{), }\AttributeTok{legend.position =} \StringTok{"bottom"}\NormalTok{)}
\FunctionTok{theme\_set}\NormalTok{(A05\_theme)}
\end{Highlighting}
\end{Shaded}

\hypertarget{create-graphs}{%
\subsection{Create graphs}\label{create-graphs}}

For numbers 4-7, create ggplot graphs and adjust aesthetics to follow
best practices for data visualization. Ensure your theme, color
palettes, axes, and additional aesthetics are edited accordingly.

\begin{enumerate}
\def\labelenumi{\arabic{enumi}.}
\setcounter{enumi}{3}
\tightlist
\item
  {[}NTL-LTER{]} Plot total phosphorus (\texttt{tp\_ug}) by phosphate
  (\texttt{po4}), with separate aesthetics for Peter and Paul lakes. Add
  a line of best fit and color it black. Adjust your axes to hide
  extreme values (hint: change the limits using \texttt{xlim()} and
  \texttt{ylim()}).
\end{enumerate}

\begin{Shaded}
\begin{Highlighting}[]
\CommentTok{\#4}
\FunctionTok{library}\NormalTok{(ggplot2)}

\NormalTok{NTL.LTER.phos.Peter }\OtherTok{\textless{}{-}} \FunctionTok{ggplot}\NormalTok{(}\FunctionTok{subset}\NormalTok{(PP\_Nutrients, lakename }\SpecialCharTok{==} \StringTok{"Peter Lake"}\NormalTok{), }\FunctionTok{aes}\NormalTok{(}\AttributeTok{x =}\NormalTok{ po4, }\AttributeTok{y =}\NormalTok{ tp\_ug)) }\SpecialCharTok{+}
  \FunctionTok{geom\_point}\NormalTok{(}\AttributeTok{color =} \StringTok{"purple"}\NormalTok{, }\AttributeTok{size =} \FloatTok{0.8}\NormalTok{, }\AttributeTok{shape =} \StringTok{"square"}\NormalTok{)}\SpecialCharTok{+}
  \FunctionTok{xlim}\NormalTok{(}\DecValTok{0}\NormalTok{, }\DecValTok{45}\NormalTok{) }\SpecialCharTok{+}
  \FunctionTok{ylim}\NormalTok{(}\DecValTok{0}\NormalTok{, }\DecValTok{150}\NormalTok{) }\SpecialCharTok{+}
  \FunctionTok{geom\_smooth}\NormalTok{(}\AttributeTok{method =}\NormalTok{ lm, }\AttributeTok{color =} \StringTok{"black"}\NormalTok{) }\SpecialCharTok{+}
  \FunctionTok{xlab}\NormalTok{(}\StringTok{"Phosphate"}\NormalTok{) }\SpecialCharTok{+}
  \FunctionTok{ylab}\NormalTok{(}\StringTok{"Phosphorus"}\NormalTok{) }\SpecialCharTok{+}
  \FunctionTok{ggtitle}\NormalTok{(}\StringTok{"Peter Lake"}\NormalTok{)}
\FunctionTok{print}\NormalTok{(NTL.LTER.phos.Peter)}
\end{Highlighting}
\end{Shaded}

\begin{verbatim}
## `geom_smooth()` using formula 'y ~ x'
\end{verbatim}

\begin{verbatim}
## Warning: Removed 11423 rows containing non-finite values (stat_smooth).
\end{verbatim}

\begin{verbatim}
## Warning: Removed 11423 rows containing missing values (geom_point).
\end{verbatim}

\includegraphics{MeilinChan_A05_DataVisualization_files/figure-latex/unnamed-chunk-3-1.pdf}

\begin{Shaded}
\begin{Highlighting}[]
\NormalTok{NTL.LTER.phos.Paul }\OtherTok{\textless{}{-}} \FunctionTok{ggplot}\NormalTok{(}\FunctionTok{subset}\NormalTok{(PP\_Nutrients, lakename }\SpecialCharTok{==} \StringTok{"Paul Lake"}\NormalTok{), }
                             \FunctionTok{aes}\NormalTok{(}\AttributeTok{x =}\NormalTok{ po4, }\AttributeTok{y =}\NormalTok{ tp\_ug)) }\SpecialCharTok{+}
  \FunctionTok{geom\_point}\NormalTok{(}\AttributeTok{color =} \StringTok{"red"}\NormalTok{, }\AttributeTok{size =} \FloatTok{0.8}\NormalTok{) }\SpecialCharTok{+}
  \FunctionTok{xlim}\NormalTok{(}\DecValTok{0}\NormalTok{, }\DecValTok{35}\NormalTok{) }\SpecialCharTok{+}
  \FunctionTok{ylim}\NormalTok{(}\DecValTok{0}\NormalTok{, }\DecValTok{150}\NormalTok{) }\SpecialCharTok{+}
  \FunctionTok{geom\_smooth}\NormalTok{(}\AttributeTok{method =}\NormalTok{ lm, }\AttributeTok{color =} \StringTok{"black"}\NormalTok{) }\SpecialCharTok{+}
  \FunctionTok{xlab}\NormalTok{(}\StringTok{"Phosphate"}\NormalTok{) }\SpecialCharTok{+}
  \FunctionTok{ylab}\NormalTok{(}\StringTok{"Phosphorus"}\NormalTok{) }\SpecialCharTok{+}
  \FunctionTok{ggtitle}\NormalTok{(}\StringTok{"Paul Lake"}\NormalTok{)}
\FunctionTok{print}\NormalTok{(NTL.LTER.phos.Paul)}
\end{Highlighting}
\end{Shaded}

\begin{verbatim}
## `geom_smooth()` using formula 'y ~ x'
\end{verbatim}

\begin{verbatim}
## Warning: Removed 10526 rows containing non-finite values (stat_smooth).
\end{verbatim}

\begin{verbatim}
## Warning: Removed 10526 rows containing missing values (geom_point).
\end{verbatim}

\includegraphics{MeilinChan_A05_DataVisualization_files/figure-latex/unnamed-chunk-3-2.pdf}

\begin{Shaded}
\begin{Highlighting}[]
\NormalTok{NTL.LTER.phos }\OtherTok{\textless{}{-}} \FunctionTok{plot\_grid}\NormalTok{(NTL.LTER.phos.Paul, NTL.LTER.phos.Peter, }
                           \AttributeTok{nrow =} \DecValTok{2}\NormalTok{, }\AttributeTok{align =} \StringTok{\textquotesingle{}hv\textquotesingle{}}\NormalTok{, }\AttributeTok{rel\_heights =} \FunctionTok{c}\NormalTok{(}\DecValTok{3}\NormalTok{,}\DecValTok{3}\NormalTok{))}
\end{Highlighting}
\end{Shaded}

\begin{verbatim}
## `geom_smooth()` using formula 'y ~ x'
\end{verbatim}

\begin{verbatim}
## Warning: Removed 10526 rows containing non-finite values (stat_smooth).

## Warning: Removed 10526 rows containing missing values (geom_point).
\end{verbatim}

\begin{verbatim}
## `geom_smooth()` using formula 'y ~ x'
\end{verbatim}

\begin{verbatim}
## Warning: Removed 11423 rows containing non-finite values (stat_smooth).
\end{verbatim}

\begin{verbatim}
## Warning: Removed 11423 rows containing missing values (geom_point).
\end{verbatim}

\begin{Shaded}
\begin{Highlighting}[]
\FunctionTok{print}\NormalTok{(NTL.LTER.phos)}
\end{Highlighting}
\end{Shaded}

\includegraphics{MeilinChan_A05_DataVisualization_files/figure-latex/unnamed-chunk-3-3.pdf}

\begin{enumerate}
\def\labelenumi{\arabic{enumi}.}
\setcounter{enumi}{4}
\tightlist
\item
  {[}NTL-LTER{]} Make three separate boxplots of (a) temperature, (b)
  TP, and (c) TN, with month as the x axis and lake as a color
  aesthetic. Then, create a cowplot that combines the three graphs. Make
  sure that only one legend is present and that graph axes are aligned.
\end{enumerate}

\begin{Shaded}
\begin{Highlighting}[]
\CommentTok{\#5}
\NormalTok{PP\_Nutrients}\SpecialCharTok{$}\NormalTok{month }\OtherTok{\textless{}{-}} \FunctionTok{as.factor}\NormalTok{(PP\_Nutrients}\SpecialCharTok{$}\NormalTok{month)}

\NormalTok{NTL.temp }\OtherTok{\textless{}{-}} \FunctionTok{ggplot}\NormalTok{(PP\_Nutrients, }\FunctionTok{aes}\NormalTok{(}\AttributeTok{x =}\NormalTok{ month, }\AttributeTok{y =}\NormalTok{ temperature\_C)) }\SpecialCharTok{+}
  \FunctionTok{geom\_boxplot}\NormalTok{(}\FunctionTok{aes}\NormalTok{(}\AttributeTok{color =}\NormalTok{ lakename)) }\SpecialCharTok{+}
  \FunctionTok{xlab}\NormalTok{(}\StringTok{"Month"}\NormalTok{) }\SpecialCharTok{+}
  \FunctionTok{ylab}\NormalTok{(}\StringTok{"Temperature (Celsius)"}\NormalTok{) }\SpecialCharTok{+}
  \FunctionTok{labs}\NormalTok{(}\AttributeTok{color =} \StringTok{"Lake Name"}\NormalTok{)}

\FunctionTok{print}\NormalTok{(NTL.temp)}
\end{Highlighting}
\end{Shaded}

\begin{verbatim}
## Warning: Removed 3566 rows containing non-finite values (stat_boxplot).
\end{verbatim}

\includegraphics{MeilinChan_A05_DataVisualization_files/figure-latex/unnamed-chunk-4-1.pdf}

\begin{Shaded}
\begin{Highlighting}[]
\NormalTok{NTL.TP }\OtherTok{\textless{}{-}} \FunctionTok{ggplot}\NormalTok{(PP\_Nutrients, }\FunctionTok{aes}\NormalTok{(}\AttributeTok{x =}\NormalTok{ month, }\AttributeTok{y =}\NormalTok{ tp\_ug)) }\SpecialCharTok{+}
  \FunctionTok{geom\_boxplot}\NormalTok{(}\FunctionTok{aes}\NormalTok{(}\AttributeTok{color =}\NormalTok{ lakename)) }\SpecialCharTok{+}
  \FunctionTok{xlab}\NormalTok{(}\StringTok{"Month"}\NormalTok{) }\SpecialCharTok{+}
  \FunctionTok{ylab}\NormalTok{(}\StringTok{"ug"}\NormalTok{) }\SpecialCharTok{+}
  \FunctionTok{xlim}\NormalTok{(}\StringTok{"5"}\NormalTok{,}\StringTok{"6"}\NormalTok{,}\StringTok{"7"}\NormalTok{,}\StringTok{"8"}\NormalTok{,}\StringTok{"9"}\NormalTok{)}

\FunctionTok{print}\NormalTok{(NTL.TP)}
\end{Highlighting}
\end{Shaded}

\begin{verbatim}
## Warning: Removed 205 rows containing missing values (stat_boxplot).
\end{verbatim}

\begin{verbatim}
## Warning: Removed 20524 rows containing non-finite values (stat_boxplot).
\end{verbatim}

\includegraphics{MeilinChan_A05_DataVisualization_files/figure-latex/unnamed-chunk-4-2.pdf}

\begin{Shaded}
\begin{Highlighting}[]
\NormalTok{NTL.TN }\OtherTok{\textless{}{-}} \FunctionTok{ggplot}\NormalTok{(PP\_Nutrients, }\FunctionTok{aes}\NormalTok{(}\AttributeTok{x =}\NormalTok{ month, }\AttributeTok{y =}\NormalTok{ tn\_ug)) }\SpecialCharTok{+}
  \FunctionTok{geom\_boxplot}\NormalTok{(}\FunctionTok{aes}\NormalTok{(}\AttributeTok{color =}\NormalTok{ lakename)) }\SpecialCharTok{+}
  \FunctionTok{xlab}\NormalTok{(}\StringTok{"Month"}\NormalTok{) }\SpecialCharTok{+}
  \FunctionTok{ylab}\NormalTok{(}\StringTok{"ug"}\NormalTok{) }\SpecialCharTok{+}
  \FunctionTok{xlim}\NormalTok{(}\StringTok{"5"}\NormalTok{,}\StringTok{"6"}\NormalTok{,}\StringTok{"7"}\NormalTok{,}\StringTok{"8"}\NormalTok{,}\StringTok{"9"}\NormalTok{)}

\FunctionTok{print}\NormalTok{(NTL.TN)}
\end{Highlighting}
\end{Shaded}

\begin{verbatim}
## Warning: Removed 205 rows containing missing values (stat_boxplot).
\end{verbatim}

\begin{verbatim}
## Warning: Removed 21378 rows containing non-finite values (stat_boxplot).
\end{verbatim}

\includegraphics{MeilinChan_A05_DataVisualization_files/figure-latex/unnamed-chunk-4-3.pdf}

\begin{Shaded}
\begin{Highlighting}[]
\NormalTok{NTL.legend }\OtherTok{\textless{}{-}} \FunctionTok{get\_legend}\NormalTok{(NTL.temp)}
\end{Highlighting}
\end{Shaded}

\begin{verbatim}
## Warning: Removed 3566 rows containing non-finite values (stat_boxplot).
\end{verbatim}

\begin{Shaded}
\begin{Highlighting}[]
\FunctionTok{print}\NormalTok{(}\FunctionTok{plot\_grid}\NormalTok{(NTL.legend)) }\CommentTok{\#only shows legend {-} no axes}
\end{Highlighting}
\end{Shaded}

\includegraphics{MeilinChan_A05_DataVisualization_files/figure-latex/unnamed-chunk-4-4.pdf}

\begin{Shaded}
\begin{Highlighting}[]
\NormalTok{NTL.three }\OtherTok{\textless{}{-}} \FunctionTok{plot\_grid}\NormalTok{(NTL.temp }\SpecialCharTok{+} \FunctionTok{theme}\NormalTok{(}\AttributeTok{legend.position =} \StringTok{"none"}\NormalTok{),}
\NormalTok{                       NTL.TP }\SpecialCharTok{+} \FunctionTok{theme}\NormalTok{(}\AttributeTok{legend.position =} \StringTok{"none"}\NormalTok{),}
\NormalTok{                       NTL.TN }\SpecialCharTok{+} \FunctionTok{theme}\NormalTok{(}\AttributeTok{legend.position =} \StringTok{"none"}\NormalTok{),}
                       \AttributeTok{nrow =} \DecValTok{1}\NormalTok{, }
                       \AttributeTok{align =} \StringTok{\textquotesingle{}hv\textquotesingle{}}\NormalTok{,}
                       \AttributeTok{labels =} \FunctionTok{c}\NormalTok{(}\StringTok{"Temp"}\NormalTok{, }\StringTok{"TP"}\NormalTok{, }\StringTok{"TN"}\NormalTok{))}
\end{Highlighting}
\end{Shaded}

\begin{verbatim}
## Warning: Removed 3566 rows containing non-finite values (stat_boxplot).
\end{verbatim}

\begin{verbatim}
## Warning: Removed 205 rows containing missing values (stat_boxplot).
\end{verbatim}

\begin{verbatim}
## Warning: Removed 20524 rows containing non-finite values (stat_boxplot).
\end{verbatim}

\begin{verbatim}
## Warning: Removed 205 rows containing missing values (stat_boxplot).
\end{verbatim}

\begin{verbatim}
## Warning: Removed 21378 rows containing non-finite values (stat_boxplot).
\end{verbatim}

\begin{Shaded}
\begin{Highlighting}[]
\FunctionTok{print}\NormalTok{(NTL.three) }
\end{Highlighting}
\end{Shaded}

\includegraphics{MeilinChan_A05_DataVisualization_files/figure-latex/unnamed-chunk-4-5.pdf}

\begin{Shaded}
\begin{Highlighting}[]
\NormalTok{NTL.three.leg }\OtherTok{\textless{}{-}} \FunctionTok{plot\_grid}\NormalTok{(NTL.three, NTL.legend,}\AttributeTok{nrow =} \DecValTok{2}\NormalTok{, }
                           \AttributeTok{rel\_heights =} \FunctionTok{c}\NormalTok{(}\DecValTok{3}\NormalTok{,}\FloatTok{0.3}\NormalTok{))}

\FunctionTok{print}\NormalTok{(NTL.three.leg)}
\end{Highlighting}
\end{Shaded}

\includegraphics{MeilinChan_A05_DataVisualization_files/figure-latex/unnamed-chunk-4-6.pdf}

Question: What do you observe about the variables of interest over
seasons and between lakes?

\begin{quote}
Answer: The temperature of both Peter Lake and Paul Lake increase as the
season shifts from winter to summer and then decreases again as the
season shifts back into autumn. TP in Peter Lake increases in ug as the
season shifts into summer and then autumn. TP in Paul Lake decreases as
the season shifts into summer but then increases again as it becomes
autumn. TN in Peter Lake increases and peaks as winter shifts into
summer (around August), but then decreases once the season starts to
roll into autumn. TN in Paul Lake seems to remain at a steady ug level
until around September when it increases.
\end{quote}

\begin{enumerate}
\def\labelenumi{\arabic{enumi}.}
\setcounter{enumi}{5}
\item
  {[}Niwot Ridge{]} Plot a subset of the litter dataset by displaying
  only the ``Needles'' functional group. Plot the dry mass of needle
  litter by date and separate by NLCD class with a color aesthetic. (no
  need to adjust the name of each land use)
\item
  {[}Niwot Ridge{]} Now, plot the same plot but with NLCD classes
  separated into three facets rather than separated by color.
\end{enumerate}

\begin{Shaded}
\begin{Highlighting}[]
\CommentTok{\#6}
\FunctionTok{class}\NormalTok{(NN\_Litter}\SpecialCharTok{$}\NormalTok{collectDate)}
\end{Highlighting}
\end{Shaded}

\begin{verbatim}
## [1] "Date"
\end{verbatim}

\begin{Shaded}
\begin{Highlighting}[]
\NormalTok{NN\_Litter}\SpecialCharTok{$}\NormalTok{collectDate }\OtherTok{\textless{}{-}} \FunctionTok{as.Date}\NormalTok{(NN\_Litter}\SpecialCharTok{$}\NormalTok{collectDate)}


\NormalTok{needles.dry.point }\OtherTok{\textless{}{-}} \FunctionTok{ggplot}\NormalTok{(}\FunctionTok{subset}\NormalTok{(NN\_Litter, functionalGroup }\SpecialCharTok{==} \StringTok{"Needles"}\NormalTok{)) }\SpecialCharTok{+}
  \FunctionTok{geom\_point}\NormalTok{(}\FunctionTok{aes}\NormalTok{(}\AttributeTok{x =}\NormalTok{ collectDate, }\AttributeTok{y =}\NormalTok{ dryMass, }
                 \AttributeTok{color =}\NormalTok{ nlcdClass), }\AttributeTok{alpha =} \FloatTok{0.7}\NormalTok{) }\SpecialCharTok{+}
  \FunctionTok{theme}\NormalTok{(}\AttributeTok{legend.position =} \StringTok{"bottom"}\NormalTok{) }\SpecialCharTok{+}
  \FunctionTok{xlab}\NormalTok{(}\StringTok{"Collection Year"}\NormalTok{) }\SpecialCharTok{+}
  \FunctionTok{ylab}\NormalTok{(}\StringTok{"Needles Dry Mass"}\NormalTok{) }\SpecialCharTok{+}
  \FunctionTok{labs}\NormalTok{(}\AttributeTok{color =} \StringTok{"NLCD Class"}\NormalTok{) }\SpecialCharTok{+}
  \FunctionTok{scale\_color\_manual}\NormalTok{(}\AttributeTok{labels =} \FunctionTok{c}\NormalTok{(}\StringTok{"Evergreen Forest"}\NormalTok{, }\StringTok{"Grassland Herbaceous"}\NormalTok{, }
                                \StringTok{"Shrub Scrub"}\NormalTok{),}
                     \AttributeTok{values =} \FunctionTok{c}\NormalTok{(}\StringTok{"dark green"}\NormalTok{, }\StringTok{"blue"}\NormalTok{, }\StringTok{"orange"}\NormalTok{))}
                      
\FunctionTok{print}\NormalTok{(needles.dry.point)}
\end{Highlighting}
\end{Shaded}

\includegraphics{MeilinChan_A05_DataVisualization_files/figure-latex/unnamed-chunk-5-1.pdf}

\begin{Shaded}
\begin{Highlighting}[]
\CommentTok{\#7}

\NormalTok{pointfacet.needles.dry }\OtherTok{\textless{}{-}} \FunctionTok{ggplot}\NormalTok{(}\FunctionTok{subset}\NormalTok{(NN\_Litter, functionalGroup }\SpecialCharTok{==} \StringTok{"Needles"}\NormalTok{)) }\SpecialCharTok{+}
  \FunctionTok{geom\_point}\NormalTok{(}\FunctionTok{aes}\NormalTok{(}\AttributeTok{x =}\NormalTok{ collectDate, }\AttributeTok{y =}\NormalTok{ dryMass, }\AttributeTok{color =}\NormalTok{ nlcdClass)) }\SpecialCharTok{+}
  \FunctionTok{theme}\NormalTok{(}\AttributeTok{legend.position =} \StringTok{"none"}\NormalTok{) }\SpecialCharTok{+}
  \FunctionTok{xlab}\NormalTok{(}\StringTok{"Collection Year"}\NormalTok{) }\SpecialCharTok{+}
  \FunctionTok{ylab}\NormalTok{(}\StringTok{"Needles Dry Mass"}\NormalTok{) }\SpecialCharTok{+}
  \FunctionTok{facet\_grid}\NormalTok{(}\SpecialCharTok{\textasciitilde{}}\NormalTok{ nlcdClass)}

\FunctionTok{print}\NormalTok{(pointfacet.needles.dry)}
\end{Highlighting}
\end{Shaded}

\includegraphics{MeilinChan_A05_DataVisualization_files/figure-latex/unnamed-chunk-5-2.pdf}
Question: Which of these plots (6 vs.~7) do you think is more effective,
and why?

\begin{quote}
Answer: Plot 7 is more effective. By separating NLCD classes into 3
facets, we are able to better see the overall spread of amount of needle
dry mass collected over the timespan of the data. From Plot 7, we can
see that across all NLCD classes, that most of the dry mass collected
over the course of the time period was below 10 units of dry mass. When
we observe Plot 6, the plot is difficult to read as some of the points
overlap with each other and it is difficult to see the full distribution
of each NLCD Class during each collection periods.
\end{quote}

\end{document}
